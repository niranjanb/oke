\eat{
1 Introduction 1
1.1 Motivation . . . . . . . . . . . . . . . . . . . . . . . . . . . . . . . . . . . . . . . . . . . . 1
1.2 Research challenges . . . . . . . . . . . . . . . . . . . . . . . . . . . . . . . . . . . . . . . 2
1.3 Contributions . . . . . . . . . . . . . . . . . . . . . . . . . . . . . . . . . . . . . . . . . . 4
2 Background 4
2.1 Reducing network energy consumption . . . . . . . . . . . . . . . . . . . . . . . . . . . . 4
2.2 Reducing the energy consumption of other device components . . . . . . . . . . . . . . . . 5
3 Power management as a network primitive 5
3.1 Preliminary work . . . . . . . . . . . . . . . . . . . . . . . . . . . . . . . . . . . . . . . . 6
3.2 Proposed work . . . . . . . . . . . . . . . . . . . . . . . . . . . . . . . . . . . . . . . . . 8
3.2.1 System architecture . . . . . . . . . . . . . . . . . . . . . . . . . . . . . . . . . . . 8
3.2.2 Interfacing with mobile device components . . . . . . . . . . . . . . . . . . . . . . 9
3.2.3 Leveraging external devices . . . . . . . . . . . . . . . . . . . . . . . . . . . . . . 10
4 User-centric power management 11
4.1 Background . . . . . . . . . . . . . . . . . . . . . . . . . . . . . . . . . . . . . . . . . . . 11
4.2 Preliminary work . . . . . . . . . . . . . . . . . . . . . . . . . . . . . . . . . . . . . . . . 11
4.3 Proposed work . . . . . . . . . . . . . . . . . . . . . . . . . . . . . . . . . . . . . . . . . 12
4.3.1 Modeling energy consumption of a user . . . . . . . . . . . . . . . . . . . . . . . . 12
4.3.2 Leveraging user-activity prediction . . . . . . . . . . . . . . . . . . . . . . . . . . 13
5 Development Plan and Timeline 14
6 Broader Impact 14
7 Curriculum Development Activities 15
8 Prior NSF Support 15
9 Data Management Plan 16
}


\section{Open-domain Background Knowledge Extraction}

%The problem becomes worse when trying to extract information about events with multiple actors performing different roles. Template-driven extraction (Template IE) uses templates for event types (e.g., arrest, arson, bombing) and extract information into slots that correspond to actors and their roles. While templates provide a rich output structure, they are fundamentally limited by the cost of manual authoring of templates. Until recently there was no analog to Open IE for template IE. Chambers was the first to show how such templates can be automatically mined from text. However, the work resulted in schemas that mix events from different domains and with a small number of actors. In our previous work we enhanced schema generation by using an Open IE solution that scale to arbitrary domains but also resulted in highly coherent schemas with many actors. 
%Macro-reading works for accumulating factual statements: e.g., nationality(Maradona, Argentina), player(Maradona, Soccer) etc. However, micro-reading is often essential to answering questions about individuals or events that are not discussed often in texts. For example, consider a snippet from a news article:

The project will proceed in three phases:
\begin{itemize}
\item Knowledge Representation
Identify and design a suitable representation for the background knowledge by analyzing manually generated background knowledge. The idea here is to design games that help elicit readers to specify the background knowledge they use in reading or understanding a document. 
Once we have aggregated the various forms of knowledge we will design a suitable representation that allows for . 

\item Extraction and Generalization -- In this phase we will 

\item Aggregation and Consolidation


\end{itemize}